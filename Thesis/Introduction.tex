% \documentclass{article}
% \usepackage{graphicx} 
% \usepackage{amsmath}  
% \usepackage{tikz}
% \usepackage[a4paper, margin=4cm]{geometry}
% \setlength{\parindent}{0pt}
% \begin{document}

\section{Introduction}

During the SARS-COV-2 pandemic, epidemic models were shown to fail in accounting for human behavior and the heterogeneous structure of the contact network, both of which significantly impact epidemic dynamics \cite{Lewis2021wrong}. Non-pharmaceutical interventions such as travel restrictions have proven essential in preventing the spread of the infectious disease. However, the efficiency and restraint in implementing such restrictions must consider economic costs and potential societal backlash due to prolonged quarantine periods.\\ 

Epidemic modeling has a rich history rooted in mathematical and computational frameworks aimed at understanding the dynamics of infectious disease spread. The foundational models typically focus on a single, homogeneously mixed population, where it is assumed that all individuals interact with each other with equal probability. At any given time, each individual belongs to a specific state with respect to the infectious disease, such as susceptible, infected, or recovered. These states are often represented in compartmental models, where individuals are aggregated into compartments based on their disease status.\\

The most basic compartmental models, such as the Susceptible- Infectious- Recovered (SIR), rely on differential equations to describe the rates at which individuals transition between compartments. Transition rates between these states are often modeled as stochastic processes, with transitions assumed to occur as a Poisson process. This implies that the time between events (e.g., an individual becoming infected or recovering) follows an exponential distribution, simplifying the mathematical treatment and enabling the calculation of expected transition probabilities over time.\\

For instance, in the SIR model, the rate at which susceptible individuals become infected depends on the interaction probability between susceptible and infectious individuals and the transmission probability per contact. These interactions are assumed to occur uniformly across the population in homogeneously mixed models. However, this assumption oversimplifies the complexity of real-world social networks, where contact patterns are highly structured and heterogeneous.\\

To address this limitation, epidemic models have been extended to incorporate heterogeneity in interactions. One such extension involves network-based models, where individuals are represented as nodes, and their interactions are represented as edges in a network. This framework allows for irregular degree distributions, capturing the reality that some individuals have many more contacts than others. These models enable a more nuanced understanding of how diseases spread through populations with diverse social structures, including scale-free networks and small-world networks. However, these kind of network-based models require either in-depth data collection to capture the intricate contact network topology or broad simplified assumptions on the network structure such.    \\

Building on these advances, hybrid models such as metapopulation networks have gained attention. The concept of metapopulations comes originally from ecology, where one or more species inhabit several independent but connected patches such as islands. The collection of populations in this group of islands is called a metapopulation, where each patch is inhabited by a subpopulation. Metapopulation models typically assumed homogeniously mixed populations in each patch with some possibility of individuals migrating from one patch to another, causing repopulation of a patch or spreading a disease. While this model is a natural extention to a group of homogeneously mixed populations, it fails to capture the heterogenous traffic or migration in between various patches. To fix this, a metapopulation network is defined such that the subpopulations are connected by a network topology, allowing individuals to migrate between them along the links at different rates. The rate of migration, often modeled as a Poisson process, defines the mobility of individuals between subpopulations. Within each subpopulation, interactions between individuals are assumed to be homogeneous, facilitating the use of compartmental aggregation. However, the inclusion of migration pathways adds complexity to the system, as the probability of interaction now depends on both intra-subpopulation dynamics and inter-subpopulation connections. Such a framework is particularly relevant for systems with high network modularity, where subpopulations correspond to distinct communities such as districts, cities, or countries. This modular structure allows researchers to model localized dynamics while accounting for the influence of external connections, such as travel or trade routes, on disease transmission.\\

Epidemic dynamics in compartmental models typically come from an amortization of stochastic processes, capturing the randomness inherent in individual-level interactions and state transitions. In a single population, individuals transition between states based on probabilistic rules. This stochastic framework allows for the detailed tracking of individual trajectories and fluctuations in disease spread. At the core of stochastic epidemic models is the assumption that transitions between states occur with rates proportional to the current number of individuals in specific compartments. For example, in a homogeneously mixed population, the probability of a susceptible individual becoming infected depends on the contact rate between susceptible and infected individuals and the likelihood of transmission upon contact. These interactions are described by event-driven processes, where:

\begin{itemize}
    \item \textbf{Infection Events:} Occur as a Poisson process with a rate dependent on the number of susceptible ($S$) and infected ($I$) individuals and the transmission rate ($\beta$).
    \item \textbf{Recovery Events:} Occur as independent Poisson processes for each infected individual, governed by a recovery rate ($\gamma$).
\end{itemize}

The stochastic nature of these transitions means that even under identical initial conditions, the trajectory of an epidemic can vary due to random fluctuations. This randomness is particularly significant in small populations or during the early stages of an outbreak, where stochastic effects can influence the probability of disease extinction or large-scale spread. As the population size increases, stochastic fluctuations average out, and the system's behavior becomes increasingly deterministic. This transition, known as the mean-field limit, is achieved by considering the expected rates of change in the compartments rather than the random transitions of individuals. In this framework, the stochastic processes governing transitions are replaced by deterministic flows based on the average interaction rates.

The mean-field approximation assumes that interactions between individuals occur uniformly across the population (in the case of a homogeneously mixed population). This assumption simplifies the stochastic process into a set of ordinary differential equations (ODEs) describing the aggregate dynamics of the system. While the stochastic model provides detailed insights into individual-level processes, the mean-field limit focuses on population-level trends, enabling a more tractable analysis of large-scale epidemic behavior.

In metapopulation networks, stochastic processes govern both the within-subpopulation dynamics (e.g., local infections and recoveries) and the between-subpopulation processes, such as migration. Migration introduces additional complexity, as individuals can move between patches at rates that may vary based on the connectivity of the network and other factors such as applied mobility restrictions.

Stochastic transitions in metapopulation networks are governed by two main processes:

\begin{enumerate}
    \item \textbf{Local Dynamics:} Within each subpopulation, stochastic transitions follow the same principles as in a single population, with local rates of infection and recovery modeled as Poisson processes.
    \item \textbf{Migration Dynamics:} Movement of individuals between subpopulations is typically modeled as a Poisson process, with rates determined by the migration patterns encoded in the network topology. These rates can be heterogeneous, reflecting variations in connectivity and travel behavior across the network.
\end{enumerate}

As with single populations, the mean-field limit can be applied to metapopulation networks by aggregating the stochastic transitions into deterministic flows. This leads to a coupled system of mean-field equations, where the dynamics of each subpopulation are influenced by both local interactions and migration-driven exchanges with neighboring subpopulations.

In this study, we introduce an extension to the existing model by incorporating adaptive mobility in the network. Subpopulations can restrict the mobility of incoming and outgoing individuals based on the prevalence of infections in incoming individuals from neighboring subpopulations.

% \end{document}