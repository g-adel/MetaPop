

\section{Model}

To model the epidemic dynamics within each subpopulation, we use a mean-field compartmental model, assuming homogeneous mixing of individuals. Specifically, we employ the SIR model (Susceptible, Infected, and Recovered/Removed), which is commonly used as a simple epidemic model. The model follows a deterministic Poisson process where individuals can transition between compartments based on predefined rules. In the SIR model, the transition rules are: \( S + I \rightarrow 2I \) (an infected individual infects a susceptible individual) and \( I \rightarrow R \) (an infected individual recovers). No immunization loss process is considered in this model, focusing on a single-wave scenario. The model for each subpopulation \( x \) is described by the following set of ordinary differential equations (ODEs):

\begin{equation}
\begin{aligned}
    \dot{S}_{x} &= -\beta_{x} S_{x} I_{x}, \\
    \dot{I}_{x} &= \beta_{x} S_{x} I_{x} - \gamma_{x} I_{x}, \\
    \dot{R}_{x} &= \gamma_{x} I_{x},
\end{aligned}
\end{equation}
where \( \beta_x \) and \( \gamma_x \) represent the infection and recovery rates at site \( x \), respectively. The term \( S_x I_x \) approximates the interaction rate between susceptible and infected individuals, which is valid for large populations. The transition rates are assumed to be uniform across all subpopulations, and each subpopulation has a total number of individuals \( N_x = S_x + I_x + R_x \).

A simplified mechanistic approach is used, assuming a Markovian process in which the movement of individuals between subpopulations is governed by a matrix \( \Delta \), where \( \Delta_{ij} \) represents the rate at which individuals move from subpopulation \( i \) to subpopulation \( j \) per unit time. Although a unique \( \Delta \) matrix could be defined for each state (S, I, or R), for simplicity, we unify the mobility for all states. The most general model is described by the following system of ODEs:

\begin{equation}
\begin{aligned}
    \dot{S}_{x} &= -\beta_{x} S_{x} I_{x} - \mu_{(S)} \sum_{x' \in \mathcal{X}} \Delta_{(S);x,x'}^{\top} S_{x'}, \\
    \dot{I}_{x} &= \beta_{x} S_{x} I_{x} - \gamma_{x} I_{x} - \mu_{(I)} \sum_{x' \in \mathcal{X}} \Delta_{(I);x,x'}^{\top} I_{x'}, \\
    \dot{R}_{x} &= \gamma_{x} I_{x} - \mu_{(R)} \sum_{x' \in \mathcal{X}} \Delta_{(R);x,x'}^{\top} R_{x'},
\end{aligned}
\end{equation}
for every \( x \in \mathcal{X} \).

We define an invariant weighted mobility network \( D \), representing the pre-epidemic traffic between subpopulations. Two subpopulations are considered neighbors if and only if the mobility network weight is non-zero. To ensure that the mobility network does not affect population sizes over time, it must be undirected, meaning the net flow of travelers between two subpopulations is equal in both directions. This condition is difficult to implement when subpopulations have heterogeneous size distributions. Our model resolves this issue by making the mobility rate between two subpopulations proportional to the product of their population sizes, resulting in equal diffusion in both directions. The base mobility rate between subpopulations is represented by a positive symmetric matrix \( W \), and the mobility network is defined as:

\begin{equation}
D_{ij} = N_i N_j W_{ij}.
\end{equation}

To incorporate adaptivity, subpopulations can restrict the mobility rate through a scalar travel restriction factor \(t\). The travel restriction factor for a connection between two subpopulations ranges from zero to one, where one eliminates all mobility and zero allows full mobility. A value of zero, however, does not guarantee a pre-pandemic mobility rate, as the other side of the link could be implementing a travel restriction. Each subpopulation can control the level of travel restriction independently for each neighboring subpopulation, forming a vector \( \mathbf{t} \) of travel restriction values. The values in the vector that would correspond to links to non-neighbors or the self-link equal zero. These vectors are combined to form a non-symmetric square matrix \( T \), where \( \Delta = T \circ T^\top \circ D \).

A more refined version of the equations for subpopulation \( i \) in vector form is:

\begin{align}
    \dot{S}_{i} &= -\beta S_{i} I_{i} - \sum_{\substack{j=1 \\ j \neq i}}^n \mathbf{t}_i \mathbf{t}_j \mathbf{d}_j S_j, \\
    \dot{I}_{i} &= \beta S_{i} I_{i} - \gamma I_{i} - \sum_{\substack{j=1 \\ j \neq i}}^n \mathbf{t}_i \mathbf{t}_j \mathbf{d}_j I_j, \\
    \dot{R}_{i} &= \gamma I_{i} - \sum_{\substack{j=1 \\ j \neq i}}^n \mathbf{t}_i \mathbf{t}_j \mathbf{d}_j R_j.
\end{align}

A few deductions can be immediately made for this model formulation. First of all, it can be easily proven that the steady state of a subpopulation under SIR is either infection-free where the infection has not been introduced or has undergone infection and has fully recovered from it. Secondly, due to the nature of diffusion, it is to be expected that once a subpopulation contains a nonzero amount of infected individuals, it immediately spreads to not only all neighboring subpopulations, but all subpopulations in the connected component or subgraph of the infected subpopulation. The combination of these two observations is crucial as it eliminates the possibility of finding a travel restriction strategy that eradicates the disease in the metapopulation without all individuals getting infected once. There are several ways to counteract this issue, one which was explored is introducing the discrete probabilistic dynamics of interacting and migrating individuals, which does have a chance for an early end of the spread \cite{colizza2008epidemic}. While it is true that preventing an epidemic  from spreading to neighboring sites has been found to be very difficult, the instantaneous spread of the disease across all connected nodes is an unrealistic phenomenon. However, since the focus of this work is to slow down the spread of the infection and not completely eradicating it, this simplification is acceptable. 

\section{Strategies}
To model the travel restrictions implemented as the prevalence of infection in the metapopulation changes over time, several strategies were explored. For each subpopulation $x$, given the control vector $\mathbf{t}$, it is desirable to minimize the mobility of infected individuals, $I_{x}$, down to zero. Since implementing travel restrictions typically takes a nonzero amount of time, giving a subpopulation the ability to immediately change the travel restriction to any desired value is unrealistic. The option of cutting off the connection to an infected population after a set time has been explored \cite{feng2020infectious}. Instead, here the subpopulations have control over the rate of change of $\mathbf{t}$, which defines how $\mathbf{t}$ changes with time. Henceforth, a strategy defines an equation that $\mathbf{\dot{t}}$ follows given the current information about the metapopulation.

\subsection{Global proportional travel restriction}
The most basic form of travel restriction for the metapopulation is one where only the global infection prevalence is used to control restriction for all connections. This global travel restriction amounts to a single scalar value $t$ that subpopulations implement, which scales down the mobility of all individuals across all subpopulation connections, from both origin and destination. This results in the equation:

\begin{equation}
\dot{t} = \lambda M_{(I)}
\end{equation}

where $M_{(I)}$ is the total mobility of infected individuals on all connections:

\begin{equation}
M_{(I)} = \sum_{i=1}^{n} \sum_{j=1}^{i-1} \mathbf{t}_{i} \mathbf{t}_{j} \mathbf{d}_{j} I_{j}
\end{equation}

Applying this global restriction on the metapopulation network, the final effective Laplacian of the network is:

\begin{equation}
\Delta_G = t^2 D
\end{equation}

\subsection{Uniform proportional travel restriction}

A slightly more granular strategy would be for each subpopulation to look at the influx of infected individuals from all neighbors and restrict all connections uniformly based on the total number of infected individuals. This gives each subpopulation a single scalar variable $t_x$ to control. This restriction follows the ODE:

\begin{equation}
\dot{t}_x = \lambda M_{(I);x}
\end{equation}

where:

\begin{equation}
M_{(I);x} = \sum_{x' \in X}^{n} \mathbf{t}_x \mathbf{t}_{x'} \mathbf{d}_{x'} I_{x'}
\end{equation}

\subsection{Connection proportional travel restriction}

Finally, a third strategy is investigated where subpopulations are given individual control over restrictions for each connection to neighbors. The strategy for a connection from subpopulation $x$ to subpopulation $x'$ is:

\begin{equation}
\dot{t}_{xx'} = \lambda M_{(I);x,x'}
\end{equation}

where:

\begin{equation}
M_{(I);x,x'} = t_x t_{x'} d_{x'} I_{x'}
\end{equation}

Unlike the other two strategies, this one is entirely local to the connection, and hence the resulting system can be classified under adaptive dynamical weighted networks\cite{berner2023adaptive}.
