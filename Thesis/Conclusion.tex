\section{Conclusion}

This thesis has explored adaptive containment strategies for epidemics in metapopulation networks, focusing on the dynamic adjustment of migration rates based on infection prevalence. The main contributions of this work include the development of an adaptive mobility model, the formulation of travel restriction strategies incorporating socio-economic components, and the validation of analytical approximations with numerical simulations.\\

The adaptive mobility model introduced in this thesis dynamically adjusts migration rates based on infection prevalence, providing a more realistic representation of human behavior during an epidemic. Various travel restriction strategies were developed and analyzed, including global, uniform, and connection proportional restrictions. These strategies were designed to balance public health and socio-economic impacts. A reaction-diffusion framework was formulated to model epidemic spread in metapopulation networks, incorporating both local infection dynamics and migration between subpopulations. The analytical approximations for infection spread rates were validated through numerical simulations, demonstrating the accuracy of the proposed models.\\

The infection spread rate in non-adaptive path graphs was accurately predicted, with numerical simulations confirming the analytical approximations. The impact of the reactivity constant on the spread rate was investigated, showing that adaptivity delays the global infection spread but does not significantly alter the overall spread rate. Different network topologies, including Watts-Strogatz and Barabasi-Albert networks, were analyzed to reflect real-life transportation networks. The Lambert W function approximation was validated for spread rates in these networks.

Various travel restriction strategies were tested, with the connection-logarithm strategy showing promise in slowing the infection spread rate for a sizable distance around the initially infected subpopulation. A mobility restoration bias was introduced to prevent prolonged network shutdown, ensuring some network activity while slowing down the spread.\\

Future work should explore directional travel restrictions to further optimize containment strategies. Investigating other local travel restriction strategies could enhance the effectiveness of adaptive models. Extending diffusion models to incorporate more complex network structures and interactions is another potential area for future research. Developing metrics that balance public health and socio-economic incentives could guide policy decisions. Further simulations with different network topologies and parameter settings would help validate and refine the proposed models.\\

In conclusion, this thesis has demonstrated the potential of adaptive containment strategies to manage epidemics in metapopulation networks effectively. By dynamically adjusting migration rates and incorporating socio-economic components, these strategies offer a promising approach to balancing public health and economic impacts during an epidemic. Future research should continue to refine these models and explore new strategies to enhance their effectiveness.